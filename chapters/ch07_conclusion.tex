%!TEX root = ../report.tex

\begin{document}
    \chapter{Conclusions}
Many aspects of the real world data are indexed by time, thus analyzing the time series data has become a essential research problem. For getting a better understanding of the real world data first and most relevant approach is exploring the domain knowledge. Problem with this method is that it cannot be generalized and requires relevant skill set in the domain. Hence unsupervised algorithms were applied to solve this problem. This work focuses on to exploring methods that can be applied for feature extraction of the real world complex industrial dataset.



    \section{Contributions}
    \begin{itemize}
    \item 	
    \end{itemize}
    
    
    \section{Lessons learned}
    \begin{itemize}
    	\item The performance is affected by imbalance in data.
    	\item Domain specific knowledge plays important role in performance of model.
    	\item Spectrogram approach is not suitable for industrial dataset in use which can be analyses from the results of both transfer learning and autoencoder based feature extraction model.
    	\item Based on the reconstruction error of anomaly detection module the features of the data that differentiate abnormal behavior from normal are not that prominent.
    	\item Domain based feature extraction is best suitable for this problem.
    \end{itemize}

    \section{Future work}
\end{document}
