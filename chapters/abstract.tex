%!TEX root = ../report.tex

\begin{document}
    \begin{abstract}
        Applying machine learning techniques to solve engineering problems in the industry is often challenging as the data available or captured is many times to find practical solutions under such circumstances. This calls for a multi-pronged strategy for using different machine learning techniques. This research project uses real-time complex industrial data and real-time data from an accelerometer to analyse the comfortable closing of the car door.  The project aims to find a suitable feature extraction approach that aids in monitoring the proper closure of the door. 
        
        As building a domain specific feature set each time a new information is obtained is a tedious task. To generalize the problem this project aims to identify automatic feature extraction methods for time series data. There are three main approaches 
        
        The results show that the handcrafted method works better in detecting negative instances from the data in comparison to deep-learning based algorithms. The main reason behind being highly imbalanced data. Also the positive and negative time series instances obtained from accelerometer are very similar.
    \end{abstract}
\end{document}
