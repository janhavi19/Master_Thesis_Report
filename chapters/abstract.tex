%!TEX root = ../report.tex

\begin{document}
    \begin{abstract}
        Applying machine learning techniques to solve engineering problems in the industry is often challenging as the data available or captured is many a times very limited or random in nature. To find practical solutions under such circumstances, calls for a multi-prolonged strategy for using different machine learning techniques. This research project uses real-time complex industrial data from an accelerometer to analyses the comfortable closing of the car door.  The project aims to find a suitable feature extraction approach that aids in monitoring the proper closure of the door. 
        
        As building a domain specific feature set each time for problems with different domains is a tedious task. To generalize the solution this project aims to identify automatic feature extraction methods for time series data. There are three main approaches : Application of pre-existing models on signal data, analyzing the data in frequency-time domain and extraction of features from the image transformed data.
        
        The results show that the handcrafted method works better in detecting negative instances from the data in comparison to deep-learning based algorithms. The main reason behind being the data analyzed is highly imbalanced data. Also the positive and negative time series instances obtained from accelerometer are very similar in structure. This makes it more challenging to extract features from the raw signal data. In this thesis an attempt is made to solve these challenges
    \end{abstract}
\end{document}
